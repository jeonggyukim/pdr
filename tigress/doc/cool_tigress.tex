%\documentclass[manuscript]{aastex}
\documentclass[iop,numberedappendix]{emulateapj}
\usepackage{natbib}
\usepackage{amsmath}
\usepackage{booktabs}
\usepackage{color}
\usepackage{graphicx, subfigure}
\citestyle{aa}
\usepackage{CJKutf8}
\interfootnotelinepenalty=10000
\newcommand{\Munan}[1]{{\color{red}#1}}
\newcommand{\di}{\mathrm{d}} 
\newcommand{\e}{\mathrm{e}} 
\newcommand{\mr}{\mathrm} 
\newcommand{\Ht}{\mathrm{H_2}} 
\newcommand{\Ho}{\mathrm{H}} 
\newcommand{\HI}{\mathrm{HI}} 
\newcommand{\Hplus}{\mathrm{H^+}}
\newcommand{\He}{\mathrm{He}} 
\newcommand{\Heplus}{\mathrm{He^+}}
\newcommand{\HCOplus}{\mathrm{HCO^+}}
\newcommand{\CO}{\mathrm{CO}} 
\newcommand{\CI}{\mathrm{CI}}
\newcommand{\OI}{\mathrm{OI}}
\newcommand{\Cplus}{\mathrm{C^+}} 
\newcommand{\CHx}{\mathrm{CH_x}} 
\newcommand{\CH}{\mathrm{CH}} 
\newcommand{\OHx}{\mathrm{OH_x}} 
\newcommand{\Oplus}{\mathrm{O^+}} 
\newcommand{\Si}{\mathrm{Si}} 
\newcommand{\Siplus}{\mathrm{Si^+}} 


\defcitealias{GO2015}{GO15}

\begin{document}
\title{Heating and cooling in TIGRESS simulations}
\author{Munan Gong (\begin{CJK*}{UTF8}{gbsn}龚慕南\end{CJK*})\altaffilmark{1}}
\altaffiltext{1}{Max-Planck Institute for Extraterrestrial Physics,
Garching by Munich, 85748, Germany; 
munan@mpe.mpg.de}

\begin{abstract}
    Heating and cooling routine for TIGRESS simulations \citep{KO2017}
    based on equilibrium chemistry in \citep{GOW2016}.
\end{abstract}

\section{Introduction}
The aim here is to develop a physically motivated accurate 
heating and cooling function
that takes the hydro and radiation variables in the TIGRESS simulation and
gives a heating and cooling rate. The heating and cooling rates given here 
are very similar to the results from equilibrium chemistry, but are
obtained without the need to solve the full chemistry ODEs. The abundances of
different species are calculated (semi-)analytically from equilibrium conditions.
The heating and cooling rates then can be calculated from analytic expressions
or interpolation tables (in the case of $\CO$ rotational lines).

Section \ref{section:para} gives a summary of the input and output parameters.
Section \ref{section:chem} explains the calculation of the
abundances of chemical species. Sections \ref{section:heating} and
\ref{section:cooling} describes the heating and cooling processes included.

\section{Input and output parameters}\label{section:para}
Table \ref{table:input} list all the input parameters. The outputs are the
heating rate $\Gamma$ and cooling rate $\Lambda$:
\begin{equation}\label{eq:heating}
    \Gamma(n_\Ho, T, Z, G_\mr{PE}) = \Gamma_\mr{PE} + \Gamma_\mr{CR} +
    \Gamma_\mr{H_2 pump},
\end{equation}
and
\begin{align}\label{eq:cooling}
    \begin{split}
    &\Lambda(n_\Ho, T, \langle |\di v/\di r| \rangle, Z, G_\mr{PE},  G_\CI,
    G_\CO(,G_\Ht)) \\
    &= \Lambda_\mr{Ly\alpha}+\Lambda_\OI + \Lambda_\Cplus +
    \Lambda_\CI + \Lambda_\CO + \Lambda_\mr{rec}.
    \end{split}
\end{align}

The details about these heating and cooling processes are explained in the
following sections.

\begin{table*}[htbp]
    \centering
    \caption{Input parameters}
    \label{table:input}
    \begin{tabular}{l l l}
        \tableline
        \tableline
        Symbol & Meaning & Note\\ 
        \tableline
        \multicolumn{3}{l}{Hydro parameters:}\\
        $n_\Ho$ & number density of hydrogen atoms & $\rho=1.4271 n_\Ho m_\Ho$
        \\
        $T$ & temperature & $T=\mu m_\Ho P/ (\rho k_b)$, $\mu$ is the molecular
        weight\tablenotemark{a} 
        \\
        $\langle |\di v/\di r| \rangle$ 
        & the mean (absolute) velocity gradient\tablenotemark{b}
        & for LVG approximation in $\CO$ cooling 
        \\
        \multicolumn{3}{l}{Radiation field strengths\tablenotemark{c}:}\\
        $G_\mr{PE}$ &photo-electric heating &
        $\gamma_\mr{PE} = 1.87$
        \\
        $G_\CI$ &radiation field for $\CI$ to $\Cplus$ photo-ionization &
        $\gamma_\CI = 3.76$
        \\
        $G_\CO$ &$\CO$ photo-dissociation (only dust
        shielding) & $\gamma_\CO = 3.88$. \Munan{Use the same $\gamma$
        for $\CI$ and $\CO$?}
        \\
        $G_\Ht$\tablenotemark{d} 
        &$\Ht$ photo-dissociation (dust- and self-
        shielding) &dust shielding: $\gamma_\Ht=4.18$,
        self-shielding from \citet{DB1996}.
        \\
        \multicolumn{3}{l}{Other parameters:}\\
        $Z$ & metallicity & Assuming the same metallicity for gas and dust
        $Z=Z_d=Z_g$
        \\
        $\xi_\Ho$ & primary cosmic-ray ionization rate per $\Ho$ atom
        & scales with star formation rate and surface density\tablenotemark{e}
        \\

        \tableline
    \end{tabular}
    \tablenotetext{1}{$\mu = [m_\mathrm{He, tot} x_\mathrm{He, tot} +
    m_\Ht x_\Ht + m_\Ho (1-x_\Ht) ]/ [m_\Ho (x_\mathrm{He,tot} + x_\Ht +
    (1-2x_\Ht) + x_\e)] = (m_\mathrm{He, tot} x_\mathrm{He, tot} + m_\Ho) /
    [m_\Ho (x_\mathrm{He,tot} + 1 - x_\Ht + x_\e)] = 
    1.4271 / (x_\mathrm{He,tot} + 1 - x_\Ht + x_\e)$. $T=P/[n_\Ho k_b
    (x_\mathrm{He,tot} + 1 - x_\Ht + x_\e)]$.}
    \tablenotetext{2}{Averaged across the six faces of each grid cell in the
    simulation.}
    \tablenotetext{3}{For one-sided slab and only consider dust shielding,
    $G_i = G_0 \exp(-\gamma_i A_V) = G_0 \exp(-\sigma_i N_\Ho)$,
    where $G_0$ is the interstellar radiation field (\Munan{The spectrum from
    star clusters might be different?}), $A_V=N_\Ho/1.87\times
    10^{-21}\mr{cm^{-2}}$, and the cross-section $\sigma_i = (\gamma_i/1.87) \times
    10^{21}~\mr{cm^{-2}}$. }
    \tablenotetext{3}{We only need to calculate $G_\Ht$ if we want to include
    heating by UV-pumping of $\Ht$ (in high radiation field and low metallcitiy
    gas) or we want to calculate $\Ht$ abundances in cloud edges including FUV
    photo dissociation (not necessary in most cases, since $\Ht$ is considered
    to be affected by non-equilibrium chemistry).}
    \tablenotetext{5}{For the solar neighborhood,
    $\xi_\Ho \approx 2\times 10^{-16}~\mr{s^{-1} H^{-1}}$. 
    $\xi_{-16} = 0.472
    \frac{\Sigma_\mathrm{SFR,-3}}{\Sigma_\mathrm{gas}/50M_\odot\mathrm{pc^{-2}}
    + 1}$, $\xi_{-16} = \xi_\Ho / 10^{-16}\mr{s^{-1} H^{-1}}$, 
    $\Sigma_\mathrm{SFR,-3} =
    \Sigma_\mathrm{SFR}/10^{-3}M_\odot\mathrm{kpc^{-2}Myr^{-1}}$.}
\end{table*}

\section{Chemical Abundances}\label{section:chem}
In order to calculate the heating and cooling rates in Equations
(\ref{eq:heating}) and (\ref{eq:cooling}), we need to know the chemical
abundances of the species listed in Table \ref{table:chem}. We explain the
calculation of the abundances of these species below. 
\begin{table*}[htbp]
    \centering
    \caption{Chemical species}
    \label{table:chem}
    \begin{tabular}{l l l}
        \tableline
        \tableline
        Species & Abundance calculation & Dependence\\ 
        \tableline
        $\Ht$ & analytic, see Section \ref{section:H2}
        & $n_\Ho$, $T$, $Z_d$, $\xi_\Ho$ (and $G_\Ht$)
        \\
        $\mr{e^-}$ & iterative, assumes $x_\e=x_\Hplus + x_\Cplus$, 
        see Section \ref{section:e} 
        & $x_\Ht$, $n_\Ho$, $T$, $Z_d$, $Z_g$, $\xi_\Ho$, $G_\mr{PE}$, $G_\CI$
        \\
        $\Cplus$ &analytic, see Section \ref{section:e} 
        & $x_\e$, $x_\Ht$, $n_\Ho$, $T$, $Z_d$, $Z_g$, 
        $\xi_\Ho$, $G_\mr{PE}$, $G_\CI$
        \\
        $\Hplus$ & analytic, see Section \ref{section:e} 
        & $x_\e$, $x_\Ht$, $x_\Cplus$, $n_\Ho$, $T$, $Z_d$,
        $\xi_\Ho$, $G_\mr{PE}$
        \\
        $\CO$ &analytic, see Section \ref{section:CO} 
        & $x_\Ht$, $x_\Cplus$, $n_\Ho$, $Z_d$, $Z_g$, 
        $\xi_\Ho$, $G_\CO$
        \\
        $\HI$ &elemental conservation & $x_\HI = 1 - 2x_\Ht - x_\Hplus$ 
        \\
        $\CI$ &elemental conservation & $x_\CI = x_\mr{C,tot} - x_\Cplus -
        x_\CO$ 
        \\
        $\OI$ &elemental conservation & $x_\OI = x_\mr{O,tot} - x_\CO$ 
        \\
        \tableline
    \end{tabular}
\end{table*}

\subsection{$\Ht$ Abundance\label{section:H2}}
The $\Ht$ abundance can be obtained from Equation (18) in \citet{GOK2018}:
\begin{equation}\label{eq:GOK2018_eq18}
    f_\Ht n_\Ho k_\mr{gr} = 1.65 f_\Ht k_\mr{CR}.
\end{equation}
The left hand side is the rate of $\Ht$ formation on dust grains. On the right
hand side, $f_\Ht k_\mr{CR}$ is the destruction of $\Ht$ by cosmic rays, and
the $1.65$ factor comes from additional channels of $\Ht$ destruction by
$\mr{H_2^+}$ and $\Ht$ formation by $\mr{H_3^+}$. If we take the
photo-dissociation of $\Ht$ by FUV radiation into account, which can be
important at the edge of the cloud especially when the cosmic ray ionization
rate is low, then Equation (\ref{eq:GOK2018_eq18}) becomes:
\begin{equation}\label{eq:H2}
    f_\Ht n_\Ho k_\mr{gr} = 1.65 f_\Ht k_\mr{CR} + f_\Ht k_\gamma,
\end{equation}
where $k_\gamma=5.7\times 10^{-11}G_\Ht~\mr{s^{-1}}$ is the photo-dissociation
rate of $\Ht$. Using $x_\Ho = 1-2x_\Ht$ and
$k_\mr{CR}=2\xi_\Ho(2.3x_\Ht+1.5x_\Ho)$, Equation (\ref{eq:H2}) can be written
as a quadratic equation for $x_\Ht$:
\begin{align}
    &a x_\Ht^2 + bx_\Ht + c = 0\\
    &a = 1.155\\
    &b=- (2.475 + 2R + \frac{k_\gamma}{2\xi_\Ho}),
    ~R=\frac{k_\mr{gr}n_\Ho}{2\xi_\Ho}\\
    &c=R,
\end{align}
and the $\Ht$ abundance $x_\Ht=(-b - \sqrt{b^2 - 4ac} )/(2a)$. If $k_\gamma=0$,
this recovers the result in \citet{GOK2018} without photo-dissociation.

\subsection{$\mr{e^{-}}$, $\Cplus$ and $\Hplus$ Abundances\label{section:e}}
\subsection{$\CO$ Abundance\label{section:CO}}

\section{Heating}\label{section:heating}

\section{Cooling}\label{section:cooling}

\bibliographystyle{apj}
\bibliography{apj-jour,thesis}
\end{document}

\citet[][hereafter \citetalias{GO2015}]{GO2015}

\begin{figure}[htbp]
\centering
\includegraphics[width=\linewidth]{somefig.pdf}
\caption{caption}
\label{fig:somefig}
\end{figure}

\begin{table*}[htbp]
    \caption{caption}
    \label{table:some table}
    \begin{tabular}{l l l l l}
        \tableline
        \tableline
        No. &Reaction &Rate coefficient\tablenotemark{a} &Notes
        &Reference\\ 
        \tableline
        \multicolumn{5}{l}{Grain-assisted reactions:}\\
        1 &$\mathrm{H + H + gr \rightarrow H_2 + gr}$ 
        &$3.0\times 10^{-17}$ & &1, 2\\
        
        \tableline
    \end{tabular}
    \tablenotetext{1}{}
\end{table*}
